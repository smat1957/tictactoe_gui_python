% latex uft-8
\documentclass[uplatex,a4paper,11pt,oneside,openany]{jsbook}
%
\usepackage[dvipdfmx]{graphicx}
\usepackage{amsmath,amssymb}
\usepackage{bm}
\usepackage{graphicx}
\usepackage{ascmac}
\usepackage{setspace}
\usepackage{here}
\usepackage{listings,jlisting} %日本語のコメントアウトをする場合jlistingが必要
%ここからソースコードの表示に関する設定
\usepackage{xcolor,comment}

\definecolor{mygreen}{rgb}{0,0.6,0}
\definecolor{mygray}{rgb}{0.5,0.5,0.5}
\definecolor{mymauve}{rgb}{0.58,0,0.82}

\begin{comment}
\lstset{
  backgroundcolor=\color{white},   % choose the background color; you must add \usepackage{color} or \usepackage{xcolor}; should come as last argument
  basicstyle=\footnotesize,        % the size of the fonts that are used for the code
  breakatwhitespace=false,         % sets if automatic breaks should only happen at whitespace
  breaklines=true,                 % sets automatic line breaking
  captionpos=b,                    % sets the caption-position to bottom
  commentstyle=\color{mygreen},    % comment style
  deletekeywords={...},            % if you want to delete keywords from the given language
  escapeinside={\%*}{*)},          % if you want to add LaTeX within your code
  extendedchars=true,              % lets you use non-ASCII characters; for 8-bits encodings only, does not work with UTF-8
  firstnumber=1000,                % start line enumeration with line 1000
  frame=single,	                   % adds a frame around the code
  keepspaces=true,                 % keeps spaces in text, useful for keeping indentation of code (possibly needs columns=flexible)
  keywordstyle=\color{blue},       % keyword style
  language=Octave,                 % the language of the code
  morekeywords={*,...},            % if you want to add more keywords to the set
  numbers=left,                    % where to put the line-numbers; possible values are (none, left, right)
  numbersep=5pt,                   % how far the line-numbers are from the code
  numberstyle=\tiny\color{mygray}, % the style that is used for the line-numbers
  rulecolor=\color{black},         % if not set, the frame-color may be changed on line-breaks within not-black text (e.g. comments (green here))
  showspaces=false,                % show spaces everywhere adding particular underscores; it overrides 'showstringspaces'
  showstringspaces=false,          % underline spaces within strings only
  showtabs=false,                  % show tabs within strings adding particular underscores
  stepnumber=2,                    % the step between two line-numbers. If it's 1, each line will be numbered
  stringstyle=\color{mymauve},     % string literal style
  tabsize=2,	                   % sets default tabsize to 2 spaces
  title=\lstname                   % show the filename of files included with \lstinputlisting; also try caption instead of title
}
\end{comment}

%ここからソースコードの表示に関する設定

\lstdefinestyle{customc}{
  belowcaptionskip=1\baselineskip,
  breaklines=true,
  numbers=left,
  frame=single,
  xleftmargin=\parindent,
  language=C,
  showstringspaces=false,
  basicstyle=\footnotesize\ttfamily,
  keywordstyle=\bfseries\color{green!40!black},
  commentstyle=\itshape\color{purple!40!black},
  identifierstyle=\color{blue},
  stringstyle=\color{orange},
}

\lstdefinestyle{custompython}{
  belowcaptionskip=1\baselineskip,
  breaklines=true,
  numbers=left,
  frame=single,
  xleftmargin=\parindent,
  language=Python,
  showstringspaces=false,
  basicstyle=\footnotesize\ttfamily,
  keywordstyle=\bfseries\color{green!40!black},
  commentstyle=\itshape\color{purple!40!black},
  identifierstyle=\color{blue},
  stringstyle=\color{orange},
}

\lstdefinestyle{customjava}{
  belowcaptionskip=1\baselineskip,
  breaklines=true,
  numbers=left,
  frame=single,
  xleftmargin=\parindent,
  language=java,
  showstringspaces=false,
  basicstyle=\footnotesize\ttfamily,
  keywordstyle=\bfseries\color{green!40!black},
  commentstyle=\itshape\color{purple!40!black},
  identifierstyle=\color{blue},
  stringstyle=\color{orange},
}

\lstdefinestyle{customasm}{
  belowcaptionskip=1\baselineskip,
  frame=L,
  xleftmargin=\parindent,
  language=[x86masm]Assembler,
  basicstyle=\footnotesize\ttfamily,
  commentstyle=\itshape\color{purple!40!black},
}

\lstset{escapechar=@,style=custompython}

\makeatletter
\def\ps@plainfoot{%
  \let\@mkboth\@gobbletwo
  \let\@oddhead\@empty
  \def\@oddfoot{\normalfont\hfil-- \thepage\ --\hfil}%
  \let\@evenhead\@empty
  \let\@evenfoot\@oddfoot}
  \let\ps@plain\ps@plainfoot
\renewcommand{\chapter}{%
  \if@openright\cleardoublepage\else\clearpage\fi
  \global\@topnum\z@
  \secdef\@chapter\@schapter}
\makeatother
%
\newcommand{\maru}[1]{{\ooalign{%
\hfil\hbox{$\bigcirc$}\hfil\crcr%
\hfil\hbox{#1}\hfil}}}
%
\setlength{\textwidth}{\fullwidth}
\setlength{\textheight}{40\baselineskip}
\addtolength{\textheight}{\topskip}
\setlength{\voffset}{-0.55in}
%
\begin{document}
% START DOCUMENT
%
% COVER
\begin{center}
  \huge \par
  \vspace{15mm}
  \huge \par
  \vspace{15mm}
  \LARGE Tic Tac Toe (GUI - Python) \par
  \vspace{100mm}
  \Large \date \par
  \vspace{15mm}
  \Large \par
  \vspace{10mm}
  \Large S.Matoike \par
  \vspace{10mm}
\end{center}
\thispagestyle{empty}
\clearpage
\addtocounter{page}{-1}
\newpage
\setcounter{tocdepth}{3}
%
\tableofcontents
%
\chapter{Pythonによる三目並べ }

三目並べは「二人零和有限確定完全情報ゲーム」と呼ばれるゲームに分類されます.
簡単に言うと「勝敗は偶然に左右されず,最善手を打てばどちらかが勝つまたは引き分けになるゲーム」のことです.
他にもリバーシ,チェッカー,チェス,将棋に囲碁などもこのゲームに分類されます.

初めに三目並べというゲームについて定義をしておきます.

三目並べ:
3×3の格子上に二人のプレイヤーが「O」「X」を配置し、自分のマークを先に3つ並べた方が勝ちです.

\section{GUI版三目並べ}

新しいプロジェクトを開始することにしましょう

まずは、コンピュータが乱数で石を置いてくる単純な方法を実現します

%\begin{center}
 % \includegraphics[width=0.4\hsize]{tictactoe.png}
%\end{center}

\subsection{Params.py}

今回は、Params.pyというファイルに、
ParamsクラスとStateクラスという、2つのクラスを定義しておきます

\begin{lstlisting}[caption=Params.py,label=prog05-1]
from enum import auto

class Params:
    HUMAN_ID = auto()
    MACHINE_ID = auto()
    EMPTY_ID = auto()

class State:
    GAME = auto()
    MACHINE_WIN = auto()
    HUMAN_WIN = auto()
    DRAW = auto()
\end{lstlisting}%

Paramsクラスには、盤面に置く石の識別情報、
HUMAN\_IDとMACHINE\_ID、及び空きスロットを識別するEMPTY\_IDの3つの識別子を持たせます

Stateクラスには、盤面の状態=ゲームの進行状況を示す定数、
即ち、GAMEはゲーム進行中の状態、
HUMAN\_WINは人が勝った状態、MACHINE\_WINはコンピュータが勝った状態、
DRAWは引き分けの状態を表す定数として保持させます

いずれのクラスも、auto()という関数を使って自動的に重複しない値を持たせることができています

auto()によって自動的に区別された値が割り振られるため、
プログラム中で変数名を参照していく以上、それらの値そのものが
どのような値なのかは知らなくても問題はありませんね

また、これらはクラス変数と呼ばれ、「クラス名.変数名」という形で使うことができて、
このクラスから導出されるオブジェクトでは共通に持っている変数になります

オブジェクト毎に持たせているオブジェクト変数「self.変数名」とは別の変数ですので、
区別して理解するようにしましょう

\subsection{Screenクラス}

GUIにするので、pygameでWindow画面を用意するためのScreenクラスを作ります

\begin{lstlisting}[caption=class Screen,label=prog05-1]
import pygame

class Screen:
    def __init__(self, wh=(600,600), bgcolor=(0,0,0)):
        self.WIDTH = wh[0]
        self.HEIGHT = wh[1]
        self.COLOR = bgcolor
        self.SIZE = (self.WIDTH, self.HEIGHT)
        self.surface = pygame.display.set_mode(self.SIZE)
        self.SLOTW = self.WIDTH//3
        self.SLOTH = self.HEIGHT//3

    def fill(self):
        self.surface.fill(self.COLOR)
        BLACK = (0,0,0)
        for i in range(1,3):
            startp = (0, self.SLOTH*i)
            endp = (self.WIDTH, self.SLOTH*i)
            pygame.draw.line(self.surface, BLACK, startp, endp)
            startp = (self.SLOTW*i, 0)
            endp = (self.SLOTW*i, self.HEIGHT)
            pygame.draw.line(self.surface, BLACK, startp, endp)

    def caption(self, str):
        pygame.display.set_caption(str)
\end{lstlisting}%

コンストラクタへの引数で、
盤面のサイズと背景色を受け取ることができますが、
特に指定しない場合にはデフォルト値が用いられる様にしています

三目並べは3x3のスロットを持つ盤面ですので、
そのための描画をfill()メソッドで行っています

caption()メソッドでは、引数で受け取った文字列をWindowのタイトルバーに表示できるようにしています

\subsection{Boardクラス}

Boardクラスは、Screenクラスを継承させるようにしていますので、
Screenクラスもプロパティやメソッドを、あたかもBoardクラスで用意したかのように使うことができます

Boardクラスのコンストラクタでは、引数で盤面のサイズ、背景色、キャプションに表示する文字列を受け取ることができるようにしていますが、
時に指定しなかった場合にデフォルト値が使われるように記述しています

Boardクラスのコンストラクタの冒頭で、Screenクラスのコンストラクタを呼び出して、Screenの初期化を実行し、
キャプションの表示、そして盤面の実態である3x3=9つのEMPTY要素を持つboardという名のリストを用意しています

また、ゲームの状態をself.stateに持たせることとし、ゲーム進行中の状態GAMEを初期状態として代入しています

\begin{lstlisting}[caption=class Board,label=prog05-1]
import pygame
from Screen import Screen
from Params import State,Params

class Board( Screen ):
    def __init__(self, wh=(300,300), color=(200,200,0), str='Tic-Tac-Toe'):
        super().__init__(wh, color)
        self.caption(str)
        self.board = [Params.EMPTY_ID for _ in range(9)]
        self.state = State.GAME

    def draw(self):
        self.fill()
        N = 3
        RADIUS = int(self.SLOTW*0.2)
        BLACK = (0,0,0)
        for ypos in range(N):
            yc = ypos * self.SLOTH + self.SLOTH // 2
            for xpos in range(N):
                xc = xpos*self.SLOTW + self.SLOTW // 2
                i = ypos * N + xpos
                if self.board[i] == Params.HUMAN_ID:
                    pygame.draw.circle(self.surface, BLACK, (xc,yc), RADIUS, 3)
                elif  self.board[i] == Params.MACHINE_ID:
                    startp = (xc - RADIUS, yc - RADIUS)
                    endp = (xc + RADIUS, yc + RADIUS)
                    pygame.draw.line(self.surface, BLACK, startp, endp, 3)
                    startp = (xc + RADIUS, yc - RADIUS)
                    endp = (xc - RADIUS, yc + RADIUS)
                    pygame.draw.line(self.surface, BLACK, startp, endp, 3)

    def winner(self):
        def scan(n1, n2, n3):
            p = self.board[n1] != 0 and \
                self.board[n1] == self.board[n2] and \
                self.board[n1] == self.board[n3]
            if p:
                if self.board[n1]==Params.HUMAN_ID:
                    self.state = State.HUMAN_WIN
                elif self.board[n1]==Params.MACHINE_ID:
                    self.state = State.MACHINE_WIN
            else:
                if not self.vacant():
                    self.state = State.DRAW
                else:
                    self.state = State.GAME
            return p
        #
        return scan(0, 1, 2) or scan(3, 4, 5) or scan(6, 7, 8) or \
               scan(0, 3, 6) or scan(1, 4, 7) or scan(2, 5, 8) or \
               scan(0, 4, 8) or scan(2, 4, 6)

    def vacant(self):
        empty = []
        for n, slot in enumerate(self.board):
            if slot == Params.EMPTY_ID:
                empty.append(n)
        return empty
\end{lstlisting}%

draw()メソッドは、盤面boardリストの状態をWindowのScreen上に描画しています

boardリストの要素がHUMAN\_IDと一致するならdraw.circle()で○を描き、
MACHINE\_IDと一致するならdraw.line()を使って×を描いています

winner()メソッドでは、ゲームの勝敗、引き分け、ゲーム進行中の判別をして、
self.stateプロパティにそれぞれの値を設定しています

winner()メソッドの中に持っているローカルなscan()関数が受け取る3つのスリット番号を見て、
boardリストの該当するスロットに入っている石が同じ識別子かどうかを判定しています

同じだったら、その石がHUMAN\_IDなのかMACHINE\_IDなのかを判別して、勝者をself.stateに設定します

違っていたら、石を置く場所が残っているかどうかを調べ、残っていないないなら引き分け、
残っているようならゲーム継続の状態だとしています

scan()が返す値がTrueなら、winnerの返す値もTrueになるので、このとき勝敗がついたという意味になります

vacant()メソッドは、CUIでも説明した空きスロットのリストを返すメソッドです

\subsection{Machineクラス}

Machineクラスのput\_stone()メソッドでは、
まず、空きスロットのリストを取得し、そのリスト内の要素を乱数で選んで、
boardリスト上のその位置に、MACHINE\_IDを代入しています

Trueを返した場合は石を普通に置いた場合、Falseを返すのは盤面に空きがなかった場合です

\begin{lstlisting}[caption=class Machine,label=prog05-1]
from random import randint
from Params import Params

class Machine():
    def __init__(self, name):
        self.name = name

    def put_stone(self, board):
        vacant = board.vacant()
        if not vacant:
            return False
        n = randint(0, len(vacant)-1)
        board.board[ vacant[n] ] = Params.MACHINE_ID
        return True
\end{lstlisting}%

\subsection{Humanクラス}

Humanクラスのput\_stone()メソッドでは、まず
空きスロットのリストを取得し、空きがないならFalseで戻ります

空きがある場合は、put\_stone()メソッドの中に、
ローカルに定義したget\_slot()関数を呼びだし、その
引数に、マウスがクリックした盤面の座標をタプルの形で渡します

get\_slot()関数は、boardやスロットのサイズを元に、
マウスがクリックしたのがどのスロット上なのかを求めて、
そのスロット番号を返しています

戻されたスロット番号が、空きスロットのリストに含まれていたなら、
boardのその場所にHUMAN\_IDを代入してTrueで戻ります

マウスでクリックしたスロット番号が、空きスロットリストに含まれていないなら、
それは、既に他の石が置かれているスロットだと言うことになりますから、
もう一度クリックでスロットを選び直して下さいという意味で、Noneを返しています

\begin{lstlisting}[caption=class Human,label=prog05-1]
from Params import Params

class Human:
    def __init__(self, name):
        self.name = name

    def put_stone(self, board, pos=(-1,-1)):
        def get_slot(xy):
            x, y = xy[0], xy[1]
            N = 3
            for ypos in range(N):
                y0 = board.SLOTH * ypos
                y1 = board.SLOTH * (ypos + 1)
                for xpos in range(N):
                    x0 = board.SLOTW * xpos
                    x1 = board.SLOTW * (xpos + 1)
                    if y0 < y < y1 and x0 < x < x1:
                        return ypos * N + xpos
            return -1
        vacant = board.vacant()
        if not vacant:
            return False
        n = get_slot(pos)
        if n in vacant:
            board.board[n] = Params.HUMAN_ID
        else:
            return None
        return True
\end{lstlisting}%

\subsection{Gameクラス}

Gameクラスのコンストラクタでは、引数turnに文字列を受け取り、それによって先手と後手を決めています

change\_tuen()メソッドは、手番を交代します

fine()メソッドは、このアプリケーションを終了させるときのメソッドで、Window上のQUITのエベントを受け取ったときに呼び出されます

キーボードイベントの取得部分で、MOUSEBUTTONUPを拾っており、
humanオブジェクトのput\_stone()メソッドにクリックした座標event.posを渡しています

judge()メソッドは、盤面の状態boardオブジェクトのstateプロパティの値を見て、
勝敗が決したのか、引き分けなのか、ゲーム思考の継続なのかを判定し、
ゲーム進行中の時だけTrueを、そうでない時はFalseを返しています

start()メソッドはゲームの進行そのもので、以下を繰り返しています

①キーイベントをチェックしてHUMANの手番ならこの中でHUMANがスロットを選ぶことになります
②盤面の状態を描画します
③MACHINEの手番なら、machineオブジェクトのput\_stone()メソッドを呼び出しています
④勝敗の判定をして、⑤画面を更新します

\begin{lstlisting}[caption=class Game,label=prog05-2]
import sys
import pygame
from pygame.locals import QUIT, MOUSEBUTTONUP
from Board import Board
from Human import Human
from Machine import Machine
from Params import State

class Game:
    def __init__(self, turn='human'):
        pygame.init()
        self.board = Board()
        self.human = Human('Taro')
        self.machine = Machine('Computer')
        self.turn = False if turn=='human' else True
        self.clock = pygame.time.Clock()
        self.FPS = 10

    def change_turn(self):
        self.turn = not self.turn

    def fine(self):
        pygame.quit()
        sys.exit()

    def key_event(self):
        for event in pygame.event.get():
            if event.type == QUIT:
                self.fine()
            elif event.type == MOUSEBUTTONUP:
                if not self.turn:
                    p = self.human.put_stone(self.board, event.pos)
                    if p:
                        self.change_turn()
                    elif p==False:
                        pass    # DRAW
                    else:
                        pass

    def judge(self):
        result = False
        if self.board.winner():
            if self.board.state == State.MACHINE_WIN:
                self.board.caption('Computer won the game!')
            elif self.board.state == State.HUMAN_WIN:
                self.board.caption('You won!')
        else:
            if self.board.state == State.DRAW:
                self.board.caption('Draw!')
            elif self.board.state == State.GAME:
                result = True
        return result

    def start(self):
        while True:
            self.key_event()
            self.board.draw()
            if self.turn:
                if self.machine.put_stone(self.board):
                    self.change_turn()
            self.judge()
            pygame.display.update()
            self.clock.tick(self.FPS)
\end{lstlisting}%

\subsection{main.py}

先手が誰なのかを、Gameクラスのコンストラクタに指示しています

\begin{lstlisting}[caption=main.py,label=prog05-3]
from Game import Game

if __name__ == '__main__':
    sente = 'human'
    game = Game( turn=sente )
    game.start()
\end{lstlisting}%

\section{MiniMax法}

ここまでコンピュータが採用してきた戦略は、「空いているスロットの中から乱数で選んだ場所に石を置く」という単純なものでした

ここでは、絶対に負けないコンピュータ、
(人が最善の手をとった場合に引き分けたとしても負けることはありません)を実現していきます

\subsection{mainの変更}

Gameクラスのインスタンスgameを生成する際、
コンストラクタへの引数で、先手は'human'か'machine'か、
コンピュータの採る戦略は乱数'random'か'minimax'かを指定するように直します

\begin{lstlisting}[caption=main.py,label=minimax00]
from Game import Game
from Params import Params

if __name__ == '__main__':
    sente = 'machine'           # 'machine' or 'human'
    senryaku = 'minimax'        # 'random' or 'minimax'
    game = Game( turn=sente, strategy=senryaku )
    game.start()
\end{lstlisting}

\subsection{Gameクラスの変更}

Gameクラスのコンストラクタにおいて、MACHINEが採る戦略を引数strategyで受け取り、
それをMachineクラスのコンストラクタに渡しています

\begin{lstlisting}[caption=Game.py,label=minimax02]
import sys
import pygame
from pygame.locals import QUIT, MOUSEBUTTONUP
from Board import Board
from Human import Human
from Machine import Machine
from Params import State

class Game:
    def __init__(self, turn='human', strategy='random'):
        pygame.init()
        self.board = Board()
        self.human = Human('Taro')
        self.machine = Machine('Computer', strategy)
        self.turn = False if turn=='human' else True
        self.clock = pygame.time.Clock()
        self.FPS = 10
    ・
    以下に変更なし
    ・
    ・
\end{lstlisting}

\subsection{Boardクラスの変更}

次の2つのメソッドは、MiniMax戦略で使うために追加します

can\_put()メソッドは、引数で受け取った番号のスロットが空いているかどうかを返します

undo()メソッドは、一旦石を置いたスロットを、元の空きスロットに戻すメソッドです

デバッグのために用意したメソッドです(完成したら不要です)

debug\_board()メソッドは、デバッグ用に現在の盤面の状態をファイル出力するメソッドです

\begin{lstlisting}[caption=Board.py,label=minimax04]
import pygame
from Screen import Screen
from Params import State,Params

class Board( Screen ):
    def __init__(self, wh=(300,300), color=(200,200,0), str='Tic-Tac-Toe'):
        ・
        変更なし
        ・
        ・
    def draw(self):
        ・
        変更なし
        ・
        ・
    def winner(self):
        ・
        変更なし
        ・
        ・
    def vacant(self):
        ・
        変更なし
        ・
        ・
    # 以下を追加
    def undo(self, n):
        self.board[n] = Params.EMPTY_ID

    def can_put(self, n):
        if self.board[n]==Params.EMPTY_ID:
            return True
        return False

    # ボードを表示する for Debug
    def debug_board(self, seq, debugstring):
        str0 = '\n('+str(seq)+')'
        tmp = []
        for i in range(9):
            if self.board[i] == Params.EMPTY_ID:
                tmp.append(' ')
            elif self.board[i] == Params.HUMAN_ID:
                tmp.append('o')
            elif self.board[i] == Params.MACHINE_ID:
                tmp.append('x')
        str1 =  '\n{0[0]}|{0[1]}|{0[2]}\t<-----\t0|1|2'\
                '\n{0[3]}|{0[4]}|{0[5]}\t<-----\t3|4|5'\
                '\n{0[6]}|{0[7]}|{0[8]}\t<-----\t6|7|8\n'.format(tmp)
        str2 = str0 + str1 + debugstring + '\n'
        with open('textfile.txt', 'a', encoding='utf-8') as f:
            f.write(str2)
\end{lstlisting}

\subsection{Machineクラスの変更}

Machineクラスのコンストラクタで、戦略が'random'か'minimax'かを受け取っています

Strategyクラスを継承しており、MiniMaxの具体的な戦略のプログラムコードは
上位クラスのStrategyに記述することにします

put\_stone()メソッドでは、
①空きスロットのリストvacantを取得し、
②空きがないなら③石を置けないのでFalseで返ります。
④strategyが'random'だった場合、
⑤乱数でvacantのインデックスnを選び、
⑥vacantのn番目にあるスロット番号に石を置きます。
⑦strategyが'minimax'だった場合、
⑧MiniMax法によるベストな手をスロット番号nで受け取り、
⑨そこに石を置きます

\begin{lstlisting}[caption=Machine.py,label=minimax06]
from random import randint
from Params import Params
from Strategy import Strategy

class Machine( Strategy ):
    def __init__(self, name, strategy):
        super().__init__()
        self.name = name
        self.strategy = strategy

    def put_stone(self, board):
        vacant = board.vacant()
        if not vacant:
            return False
        if self.strategy=='random':
            n = randint(0, len(vacant)-1)
            board.board[ vacant[n] ] = Params.MACHINE_ID
        elif self.strategy=='minimax':
            n = self.bestMove(board, vacant)
            board.board[ n ] = Params.MACHINE_ID
        return True
\end{lstlisting}

\subsection{Strategyクラスを追加}

ここにMiniMax戦略のコードをまとめて記述します

bestMove()メソッドでは、
①引数で空のスロット番号のリストvacantを受け取り、
②そのリストの要素の全てについて
③順に試して
④その手を評価していきます(MaximizerであるMachineが③で石を置いた直後ですから、次はHuman即ちMinimizerの手番ですよ、という意味でFalseを引数に指示しています)
⑤高い評価値の得られたスロット番号は、
⑥bestMoveの候補として更新していきます
⑦③で選んだ手を元の空のスロットに戻して、
③次の手を置いてみて④評価することを繰り返します

minimax() メソッドでは、
①まず、boardインスタンスのwinner()メソッドを呼んで現在の盤面の状態が、
ゲーム進行中(GAME)なのか、勝敗が決した(MACHINE\_WINかHUMAN\_WIN)のか、引き分け(DRAW)だったのかを判定します
②ゲーム進行中(GAME)でないなら、③評価関数であるローカルなevaluate()メソッドを呼びだし、結果を返します
④ゲーム進行中ならば、⑤全てのスロット番号を順に選んで、⑥そのスロットが空で石を置けるかどうかを判定し、置けるならば、
⑦そのスロットに石を置きます(Maximizer=Machine='×'か、Minimizer=Human='○'かによって置く石は違います)
⑧再帰的にminimax()メソッドを呼び出します(isMaximizingPlayerをnotとして引数に指示し、
ターンを交互に割り当てるようにしています)⑨bestValを更新していきます。⑩石を置いた番号のスロットを空に戻します

minimax()メソッドの中に定義された評価関数のevaluate()は、
ゲームが進行状態(GAME)でない場合に呼び出されるので、
Machineが勝った場合(MACHINE\_WIN)、Humanが勝った場合(HUMAN\_WIN)、
引き分けだった場合(DRAW)に、それぞれ評価値を返しています


\begin{lstlisting}[caption=Strategy.py,label=minimax07]
from Params import Params, State

class Strategy:
    INFINITY = 10000
    def __init__(self):
        pass

    def bestMove(self, board, vacant):
        bestEval = -Strategy.INFINITY
        bestMove = -1
        self.debug_seq = 0
        for n in vacant:
            board.board[n] = Params.MACHINE_ID
            eval = self.minimax(board, 0, False)
            if bestEval<eval:
                bestEval = eval
                bestMove = n
            board.undo(n)
        return bestMove

    def minimax(self, board, depth, isMaximizingPlayer):
        def evaluate(depth):
            if board.state == State.MACHINE_WIN:
                board.state = State.GAME
                return 10-depth     #return 10
            elif board.state == State.HUMAN_WIN:
                board.state = State.GAME
                return depth-10     #return -10
            else:
                self.state = State.GAME
                return 0

        board.winner()
        if board.state != State.GAME:
            return evaluate(depth)
        bestVal = -Strategy.INFINITY if isMaximizingPlayer else +Strategy.INFINITY
        for n in range(9):
            if board.can_put(n):
                board.board[n] = Params.MACHINE_ID if isMaximizingPlayer else Params.HUMAN_ID
                value = self.minimax(board, depth + 1, not isMaximizingPlayer)
                bestVal = max(value, bestVal) if isMaximizingPlayer else min(value, bestVal)
                board.undo(n)
        return bestVal
\end{lstlisting}

\section{Alpha Bata Pruning}

Alpha Beta Pruningは新しい手法のアルゴリズムではなく、MiniMax法を最適化したものにすぎません

MiniMax法は、勝敗が決着するまで全ての手を試行して、その上で最善手を選ぶため、
引き分けることはあっても負けることはありません。

しかし、全ての手を試行するというのは、多くのゲームで現実的な方法にはなりません。
三目並べの様な小規模のゲームでは、それほど問題になりませんが、リバーシなどになると
思考時間がかかりすぎて実用的ではありません

そこで、MiniMax法の試行を途中で合理的に打ち切る手法がアルファ刈りベータ刈りと呼ばれる手法です

\subsection{mainの変更}

Gameクラスのインスタンスgameを生成する際、
コンストラクタへの引数で、先手は'human'か'machine'か、
コンピュータの採る戦略は乱数'random'か'minimax'か'alphabeta'の何れかを指定するようにします

\begin{lstlisting}[caption=main.py,label=minimax00]
from Game import Game
from Params import Params

if __name__ == '__main__':
    sente = 'human'           # 'machine' or 'human'
    senryaku = 'alphabeta'        # 'random', 'minimax' or 'alphabeta'
    game = Game( turn=sente, strategy=senryaku )
    game.start()
\end{lstlisting}

\subsection{Machineクラスの変更}

戦略として、alphabetaを選べるように直します

\begin{lstlisting}[caption=class Machine,label=minimax00]
from random import randint
from Params import Params
from Strategy import Strategy

class Machine( Strategy ):
    def __init__(self, name, strategy):
        super().__init__()
        self.name = name
        self.strategy = strategy

    def put_stone(self, board):
        vacant = board.vacant()
        if not vacant:
            return False
        if self.strategy=='random':
            n = randint(0, len(vacant)-1)
            board.board[ vacant[n] ] = Params.MACHINE_ID
        elif self.strategy=='minimax':
            n = self.bestMove(board, vacant)
            board.board[ n ] = Params.MACHINE_ID
        elif self.strategy=='alphabeta':
            n = self.bestMoveAB(board, vacant)
            board.board[ n ] = Params.MACHINE_ID
        return True
\end{lstlisting}

\subsection{Strategyクラスの変更}

具体的な戦略は、こちらのクラスに記述しています

\begin{lstlisting}[caption=class Strategy,label=minimax00]
from Params import Params, State

class Strategy:
    INFINITY = 10000
    def __init__(self):
        pass

    def bestMove(self, board, vacant):
            ・
            変更なし
            ・
            ・

    def minimax(self, board, depth, isMaximizingPlayer):
            ・
            変更なし
            ・
            ・

    def bestMoveAB(self, board, vacant):
        bestEval = -Strategy.INFINITY
        bestMove = -1
        for n in vacant:
            board.board[n] = Params.MACHINE_ID
            eval = self.minimaxab(board, 0, 0, False, -Strategy.INFINITY, +Strategy.INFINITY)
            if bestEval<eval:
                bestEval = eval
                bestMove = n
            board.undo(n)
        return bestMove

    def minimaxab(self, board, node, depth, isMaximizingPlayer, alpha, beta):
        def evaluate(depth):
            if board.state == State.MACHINE_WIN:
                board.state = State.GAME
                return 10-depth     #return 10
            elif board.state == State.HUMAN_WIN:
                board.state = State.GAME
                return depth-10     #return -10
            else:
                self.state = State.GAME
                return 0

        board.winner()
        if board.state != State.GAME:
            return evaluate(depth)

        bestVal = -Strategy.INFINITY if isMaximizingPlayer else +Strategy.INFINITY
        for n in range(9):
            if board.can_put(n):
                board.board[n] = Params.MACHINE_ID if isMaximizingPlayer else Params.HUMAN_ID
                value = self.minimaxab(board, node+1, depth+1, not isMaximizingPlayer, alpha, beta)
                board.undo(n)
                if isMaximizingPlayer:
                    bestVal = max(value, bestVal)
                    alpha = max(alpha, bestVal)
                else:
                    bestVal = min(value, bestVal)
                    beta = min(beta, bestVal)
                if beta<=alpha:
                    #print('depth=',depth)
                    break
        return bestVal
\end{lstlisting}




%\section*{謝辞}
%\addcontentsline{toc}{chapter}{謝辞}
%
\begin{thebibliography}{99}
  \bibitem{1}
\end{thebibliography}
%
% END DOCUMENT
\end{document}
%
